\documentclass{article}
\usepackage{amsmath}
\usepackage{amssymb}  % Required for \mathbb{N}

\title{Mathematical Proof of Correctness for Sum of Multiples of 3 or 5 Using the Inclusion-Exclusion Principle}
\author{}
\date{}

\begin{document}

\maketitle

\section*{Problem Statement}

Given a positive integer \( N \), find the sum of all natural numbers less than \( N \) that are multiples of 3 or 5.

\section*{Understanding the Inclusion-Exclusion Principle}

The inclusion-exclusion principle is a fundamental concept in combinatorics used to calculate the size or sum of the union of overlapping sets. It corrects for overcounting elements that are common to multiple sets.

\subsection*{General Formulation}

For two finite sets \( A \) and \( B \):

\[
|A \cup B| = |A| + |B| - |A \cap B|
\]

\begin{itemize}
    \item \( |A| \): Number of elements in set \( A \).
    \item \( |B| \): Number of elements in set \( B \).
    \item \( |A \cap B| \): Number of elements common to both \( A \) and \( B \).
\end{itemize}

This formula ensures that elements counted in both \( A \) and \( B \) are not double-counted in \( A \cup B \).

\subsection*{Application to Sums}

When dealing with sums of elements in sets, the principle extends as:

\[
\sum_{x \in A \cup B} x = \sum_{x \in A} x + \sum_{x \in B} x - \sum_{x \in A \cap B} x
\]

\subsection*{Why the Principle is Necessary}

If we simply add \( \sum_{x \in A} x \) and \( \sum_{x \in B} x \), any elements that are in both \( A \) and \( B \) (i.e., \( A \cap B \)) will be counted twice. The inclusion-exclusion principle corrects this by subtracting \( \sum_{x \in A \cap B} x \), ensuring each element is counted exactly once.

\subsection*{Aspect Utilized in This Context}

In the context of our problem:

\begin{itemize}
    \item \( A \): Multiples of 3 less than \( N \).
    \item \( B \): Multiples of 5 less than \( N \).
    \item \( A \cap B \): Multiples of both 3 and 5 (i.e., multiples of 15) less than \( N \).
\end{itemize}

By applying the inclusion-exclusion principle to sums, we can accurately compute the total sum of multiples of 3 or 5 below \( N \) without double-counting the multiples of 15.

\section*{Proof of Correctness}

\subsection*{Definition of Sets}

Define the sets as:

\[
A = \{ n \in \mathbb{N} \mid 1 \leq n < N,\; 3 \mid n \}
\]

\[
B = \{ n \in \mathbb{N} \mid 1 \leq n < N,\; 5 \mid n \}
\]

\[
A \cap B = \{ n \in \mathbb{N} \mid 1 \leq n < N,\; 15 \mid n \}
\]

\subsection*{Calculating the Sums}

\textbf{Multiples of 3:}

\begin{itemize}
    \item Number of terms:
\end{itemize}

\[
n_3 = \left\lfloor \frac{N - 1}{3} \right\rfloor
\]

\begin{itemize}
    \item Sum:
\end{itemize}

\[
S_3 = 3 \times \frac{n_3 (n_3 + 1)}{2}
\]

\textbf{Multiples of 5:}

\begin{itemize}
    \item Number of terms:
\end{itemize}

\[
n_5 = \left\lfloor \frac{N - 1}{5} \right\rfloor
\]

\begin{itemize}
    \item Sum:
\end{itemize}

\[
S_5 = 5 \times \frac{n_5 (n_5 + 1)}{2}
\]

\textbf{Multiples of 15:}

\begin{itemize}
    \item Number of terms:
\end{itemize}

\[
n_{15} = \left\lfloor \frac{N - 1}{15} \right\rfloor
\]

\begin{itemize}
    \item Sum:
\end{itemize}

\[
S_{15} = 15 \times \frac{n_{15} (n_{15} + 1)}{2}
\]

\subsection*{Applying the Inclusion-Exclusion Principle}

Using the principle:

\[
S = S_3 + S_5 - S_{15}
\]

This equation ensures that the sum \( S \) includes all multiples of 3 or 5 below \( N \) exactly once.

\subsection*{Explanation in Context}

\begin{itemize}
    \item \textbf{Overcounting Issue:} Multiples of 15 are included in both \( S_3 \) and \( S_5 \) because they are divisible by both 3 and 5.
    \item \textbf{Correction:} Subtracting \( S_{15} \) removes the duplicated sums of the multiples of 15, correcting the overcounting.
\end{itemize}

\subsection*{Example with \( N = 1000 \)}

\textbf{Calculations:}

\begin{itemize}
    \item \( n_3 = 333 \)
    \item \( S_3 = 166833 \)
    \item \( n_5 = 199 \)
    \item \( S_5 = 99500 \)
    \item \( n_{15} = 66 \)
    \item \( S_{15} = 33165 \)
\end{itemize}

\textbf{Final Sum:}

\[
S = 166833 + 99500 - 33165 = 233168
\]

\section*{Conclusion}

By applying the inclusion-exclusion principle, we have accurately calculated the sum of all natural numbers less than \( 1000 \) that are multiples of 3 or 5:

\[
\boxed{233168}
\]

This confirms the correctness of our function \texttt{calculate\_sum\_of\_multiples} \( (N = 1000) \).

\end{document}