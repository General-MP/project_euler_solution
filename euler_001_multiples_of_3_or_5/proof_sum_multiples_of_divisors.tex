\documentclass{article}
\usepackage{amsmath, amssymb}

\title{A General Approach to Computing the Sum of Multiples of Integers Below a Given Upper Limit}
\author{}
\date{}

\begin{document}

\maketitle

\section*{Overview and Problem Statement}
Given a set of positive integers \( D = \{d_1, d_2, \ldots, d_k\} \), our objective is to compute the sum of all natural numbers less than a positive integer \( N \) that are divisible by at least one element of the set \( D \). 

For example, if \( D = \{3, 5\} \) and \( N = 10 \), the numbers divisible by 3 or 5 are \( 3, 5, 6, 9 \). The sum of these numbers is \( 3 + 5 + 6 + 9 = 23 \).

The challenge is to generalize this process for any set \( D \) of divisors and any upper limit \( N \), ensuring no overlaps are double-counted. To accomplish this, we apply the \textbf{Principle of Inclusion-Exclusion} (PIE).

\section*{Notation and Definitions}
\begin{itemize}
    \item \( \mathbb{N} \): The set of natural numbers, defined as \( \mathbb{N} = \{1, 2, 3, \ldots\} \).
    \item \( N \in \mathbb{N} \): The positive integer upper limit (exclusive), meaning we are only interested in numbers \( < N \).
    \item \( D = \{d_1, d_2, \ldots, d_k\} \): A set of positive divisors, with each \( d_i \in \mathbb{N} \).
    \item \( \text{Sum}(d_i, N) \): The sum of all positive integers less than \( N \) that are divisible by \( d_i \).
    \item \( \text{lcm}(d_{i_1}, \ldots, d_{i_j}) \): The least common multiple of the given set of divisors.
    \item \( \left\lfloor x \right\rfloor \): The floor function, which returns the greatest integer less than or equal to \( x \).
\end{itemize}

\textbf{Additional Notation:}
\begin{itemize}
    \item \( \sum_{1 \leq i < j \leq k} \): Summation over all distinct pairs \( (i, j) \) such that \( i < j \).
    \item \( \sum_{1 \leq i < j < \ell \leq k} \): Summation over all distinct triplets \( (i, j, \ell) \) such that \( i < j < \ell \). This ensures that each combination of three distinct divisors is counted exactly once.
\end{itemize}

\section*{Theorem and Proof}
\textbf{Theorem:}  
The sum of all natural numbers less than \( N \) that are divisible by at least one of the integers in the set \( D = \{d_1, d_2, \ldots, d_k\} \) is given by:
\[
S(N, D) = \sum_{\emptyset \neq S \subseteq D} (-1)^{|S| + 1} \cdot \text{Sum}(\text{lcm}(S), N).
\]
This formula follows from the \textbf{Principle of Inclusion-Exclusion} (PIE), which ensures that elements divisible by multiple divisors are not over-counted.

\subsection*{Step 1: Sum of Multiples of a Single Divisor}
Let \( d \in \mathbb{N} \) be a divisor. The set of all natural numbers divisible by \( d \) and strictly less than \( N \) is given by:
\[
\{d, 2d, 3d, \ldots, m \cdot d\},
\]
where \( m \) is the largest positive integer such that \( m \cdot d < N \). Formally:
\[
m = \left\lfloor \frac{N - 1}{d} \right\rfloor.
\]
Thus, the numbers divisible by \( d \) below \( N \) form an \textbf{arithmetic progression} with the first term \( d \) and common difference \( d \). The general formula for the sum of the first \( m \) terms of an arithmetic progression with first term \( a \) and common difference \( d \) is:
\[
\text{Sum}(d, N) = d \cdot \frac{m (m + 1)}{2}.
\]

\subsection*{Step 2: Inclusion-Exclusion for Multiple Divisors}
When we are dealing with multiple divisors, we need to ensure that numbers divisible by more than one divisor are not counted multiple times. For this, we use the \textbf{Principle of Inclusion-Exclusion (PIE)}.

Let \( S \subseteq D \) be a non-empty subset of the divisors. The set of numbers divisible by all elements of \( S \) is precisely the set of numbers divisible by \( \text{lcm}(S) \). Therefore:
\[
\text{Sum}(\text{lcm}(S), N) = \text{Sum}\left(\text{lcm}(d_{i_1}, \ldots, d_{i_j}), N\right),
\]
where \( j = |S| \) is the size of the subset.

\[
S(N, D) =
\sum_{i=1}^{k} \text{Sum}(d_i, N)
\]

\[
- \sum_{1 \leq i < j \leq k} \text{Sum}(\text{lcm}(d_i, d_j), N)
\]

\[
+ \sum_{1 \leq i < j < \ell \leq k} \text{Sum}(\text{lcm}(d_i, d_j, d_\ell), N) 
\]

\[
- \cdots
\]

In general:
\[
S(N, D) = \sum_{\emptyset \neq S \subseteq D} (-1)^{|S| + 1} \cdot \text{Sum}(\text{lcm}(S), N).
\]

\subsection*{Explanation of Summation Notation}
\[
\textbf{Summation} \quad \sum_{1 \leq i < j \leq k}: \quad \text{This summation counts over all pairs} \quad (i, j) \quad \text{such that} \quad i < j. 
\]
This ensures that each distinct pair is considered only once.

\[
\textbf{Summation} \quad \sum_{1 \leq i < j < \ell \leq k}: \quad \text{This summation counts over all distinct triplets} \quad (i, j, \ell) \quad \text{such that} \quad i < j < \ell.
\]
This ensures that each unique triplet of divisors is included exactly once in the computation.

\subsection*{Step 3: Example with \( D = \{3, 5\} \) and \( N = 1000 \)}
Let \( D = \{3, 5\} \) and \( N = 1000 \). We compute:
\[
\text{Sum}(3, 1000) = 3 \cdot \frac{333 \cdot 334}{2} = 166833,
\]
\[
\text{Sum}(5, 1000) = 5 \cdot \frac{199 \cdot 200}{2} = 99500,
\]
\[
\text{Sum}(15, 1000) = 15 \cdot \frac{66 \cdot 67}{2} = 33165.
\]

Applying PIE:
\[
S(1000, \{3, 5\}) = 166833 + 99500 - 33165 = 233168.
\]

\section*{Conclusion}
The general formula for the sum of multiples of any combination of positive integers below a given upper limit has been rigorously derived using the principle of inclusion-exclusion. The use of ordered summations ensures correct counting of intersections without redundancy, and the methodology can be applied to any set of divisors \( D \) and any upper limit \( N \).

\end{document}
